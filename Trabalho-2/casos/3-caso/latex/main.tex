\subsection{O Método Beam-Warming}

O método de Beam-Warming pode ser considerado um método do tipo Upwind de segunda ordem. Para esse método, os fluxos são determinados mediante a adição de um termo extra aos fluxos do método Upwind de primeira ordem:

\begin{equation}
F_{i+1/2}^n = \bar{u} Q_i^n + \bar{u} \left(1 - \frac{\bar{u} \Delta t}{2 \Delta x}\right) (Q_i^n - Q_{i-1}^n)
\end{equation}

\begin{equation}
F_{i-1/2}^n = \bar{u} Q_{i-1}^n + \bar{u} \left(1 - \frac{\bar{u} \Delta t}{2 \Delta x}\right) (Q_{i-1}^n - Q_{i-2}^n)
\end{equation}

Portanto, a forma geral da atualização de \( Q_i^{n+1} \) para o método Beam-Warming é dada por:

\begin{equation}
Q_i^{n+1} = Q_i^n - C(Q_i^n - Q_{i-1}^n) - C(1 - C)(Q_i^n - 2 Q_{i-1}^n + Q_{i-2}^n)
\end{equation}

\begin{figure}[H]
    \centering
    \includegraphics[width=\textwidth]{code/images/Beam-Warming.png}
    \caption{Solução Beam-Warming para $t=1$, $t=3$, e $t=5$, com a condição inicial representada pela linha tracejada.}
\end{figure}

\begin{table}[H]
    \centering
    \input{code/tables/Beam-Warming.tex}
    \caption{Tabela de resultados para o método Beam-Warming nas posições espaciais selecionadas e diferentes tempos}
    \label{tab:beam_warming}
\end{table}

\subsection{Análise dos Resultados do Método Beam-Warming}

O método Beam-Warming é uma extensão de segunda ordem do método Upwind e oferece maior precisão em relação à dissipação. Na Figura, podemos ver que a solução para \( t = 1 \) preserva bem o perfil inicial, com uma redução limitada na amplitude. Para \( t = 3 \) e \( t = 5 \), entretanto, começam a surgir pequenas oscilações em torno das regiões de transição, o que é característico de métodos de segunda ordem sem controle de oscilação. No geral, o método Beam-Warming apresenta uma boa conservação do perfil, mas com leve tendência a oscilações em tempos maiores.

\subsection{Implementação em Python}

O código em Python é utilizado para resolver a advecção com o método Beam-Warming, aplicando condições de contorno periódicas. O código é estruturado com uma função principal \texttt{resolverAdveccao} para calcular a solução da advecção para diferentes métodos numéricos e uma função específica para o método Beam-Warming.

\begin{lstlisting}[language=Python, caption={Código para resolver a advecção usando o método Beam-Warming}, label={lst:codigo_beam_warming}]
# Função para resolver a advecção com diferentes métodos numéricos
def resolverAdveccao(metodo, condicaoInicial, intervaloTempo, intervaloEspacial, numeroCourant, tempoFinal):
    """
    Calcula a solução da advecção para um determinado método e tempo final.
    """
    densidade = condicaoInicial.copy()
    tempoAtual = 0
    while tempoAtual < tempoFinal:
        densidade = metodo(densidade, intervaloTempo, intervaloEspacial, numeroCourant)
        tempoAtual += intervaloTempo
    return densidade

# Método Beam-Warming com condições de contorno periódicas
def metodoBeamWarming(densidade, intervaloTempo, intervaloEspacial, numeroCourant):
    """
    Calcula a solução de advecção usando o método Beam-Warming.
    """
    novaDensidade = densidade.copy()
    for i in range(numPontosEspaco):
        novaDensidade[i] = densidade[i] - numeroCourant * (densidade[i] - densidade[i-1]) - \
                            0.5 * numeroCourant * (1 - numeroCourant) * (densidade[i] - 2 * densidade[i-1] + densidade[i-2])
    return novaDensidade

# Cálculo da densidade para diferentes tempos
densidadeBeamWarming1 = resolverAdveccao(metodoBeamWarming, condicaoInicial, intervaloTempo, intervaloEspacial, numeroCourant, tempoFinal1)
densidadeBeamWarming3 = resolverAdveccao(metodoBeamWarming, condicaoInicial, intervaloTempo, intervaloEspacial, numeroCourant, tempoFinal3)
densidadeBeamWarming5 = resolverAdveccao(metodoBeamWarming, condicaoInicial, intervaloTempo, intervaloEspacial, numeroCourant, tempoFinal5)
\end{lstlisting}

A função \texttt{resolverAdveccao} calcula a evolução da densidade ao longo do tempo até atingir o tempo final especificado. A função \texttt{metodoBeamWarming} implementa o método Beam-Warming para calcular a nova densidade em cada ponto do espaço, utilizando o número de Courant especificado.
