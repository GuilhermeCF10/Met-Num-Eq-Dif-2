\begin{titlepage}
    \thispagestyle{empty} % Remove números de página
    \setstretch{1.5} % Espaçamento entre linhas, certifique-se de que o pacote setspace está incluído em document.tex

    \noindent\textbf{\Large Materiais Utilizados}

    \vspace{1cm} % Espaço vertical

    Esta seção detalha os materiais e as configurações de software e hardware utilizados para o estudo e a simulação de equações diferenciais de advecção e advecção-difusão. Utilizando Python versão 3.11, este trabalho foca na implementação de modelos matemáticos e na análise de resposta do sistema sob diversas condições, simulados no ambiente do Ubuntu 24.04 LTS.

    \vspace{0.3cm} % Espaço vertical

    \textbf{Especificações do Computador e Software:}

    As simulações foram realizadas em um computador modelo Lenovo Legion 5 15IMH05H, equipado com um processador Intel® Core™ i7-10750H, 16,0 GiB de memória RAM e um disco de 512,1 GB. O sistema operacional Ubuntu 24.04 LTS foi escolhido pela sua estabilidade e compatibilidade com aplicações de engenharia.

    \vspace{0.3cm} % Espaço vertical

    \textbf{Versão do Python e Bibliotecas Utilizadas:}

    Python versão 3.11 foi utilizado para todas as simulações. As bibliotecas utilizadas incluem:
    \begin{itemize}
        \item \textbf{NumPy}: Utilizado para manipulação de arrays e operações matemáticas.
        \item \textbf{Matplotlib}: Empregada para a criação de gráficos e visualizações de dados.
        \item \textbf{SciPy}: Utilizado, especificamente o módulo \textit{erf}, para cálculos relacionados com funções especiais necessárias nas soluções das equações.
        \item \textbf{Pandas}: Utilizado para manipulação de dados e geração de tabelas.
        \item \textbf{Jinja2}: Empregado para a formatação de saída de dados, facilitando a criação de relatórios e documentos.
    \end{itemize}

\end{titlepage}
