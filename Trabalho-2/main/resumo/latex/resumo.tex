\begin{titlepage}
    \thispagestyle{empty} % Remove números de página
    \setstretch{1.5} % Espaçamento entre linhas, certifique-se de que o pacote setspace está incluído em document.tex

    \begin{center}
        \textbf{\Large RESUMO}
    \end{center}

    \vspace{1cm} % Espaço vertical

    \noindent CAGIDE FIALHO, G. Relatório do projeto de Métodos Numéricos para Equações Diferenciais II. 2024. 16 f. Trabalho da Disciplina Métodos Numéricos para Equações Diferenciais II (Graduação em Engenharia da Computação) – Graduação em Engenharia da Computação, Universidade do Estado do Rio de Janeiro, Nova Friburgo, 2024.

    \vspace{0.4cm} % Espaço vertical

    Este trabalho analisa soluções numéricas e analíticas das equações de advecção e advecção-difusão em um domínio, considerando a influência dos parâmetros de velocidade e coeficiente de difusão no transporte e dispersão de substâncias. As soluções analíticas foram obtidas com métodos de características e transformações de variáveis, enquanto as soluções numéricas foram desenvolvidas usando métodos como Upwind, Lax-Wendroff, Beam-Warming e Fromm. Os resultados das simulações numéricas destacam as diferenças entre os modelos de advecção pura e advecção-difusão, demonstrando como a difusão contribui para uma maior homogeneização da concentração ao longo do tempo. A análise comparativa das soluções revela a importância do coeficiente de difusão no comportamento de dispersão do soluto, servindo como uma ferramenta para a modelagem de fenômenos naturais e aplicações práticas em engenharia e ciências ambientais.

    \vspace{0.4cm} % Espaço vertical

    \textbf{Palavras-chave}: Métodos Numéricos, Equação de Advecção, Equação de Advecção-Difusão, Características, Simulação Numérica, Dispersão de Soluto.
\end{titlepage}
