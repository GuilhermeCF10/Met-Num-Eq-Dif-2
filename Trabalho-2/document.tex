% ===============================================================
% Document Class ================================================
\documentclass{article}

% ===============================================================
% Graphic Packages ==============================================
\usepackage{graphicx}       % Permite a inclusão de imagens no documento.
\usepackage{tocloft} % Pacote para customização de listas de conteúdos, figuras e tabelas
\usepackage{tikz}           % Ferramenta poderosa para criar gráficos programaticamente dentro do LaTeX.
\usetikzlibrary{calc}       % Extensão da biblioteca TikZ que permite cálculos mais complexos de coordenadas.
\usepackage[portuguese]{babel}
\usepackage{tocloft} % Pacote para customização de listas
\renewcommand{\listfigurename}{Lista de Figuras} % Altera o título para português
\AtBeginDocument{\renewcommand{\contentsname}{Sumário}} % Altera nome de conteudo para sumário
% ===============================================================
% Mathematical Tools ============================================
\usepackage{amsmath}        % Melhora a aparência e a flexibilidade de comandos matemáticos.
\usepackage{siunitx}        % Facilita o uso de unidades do Sistema Internacional e ajuda a formatar números complexos.

% ===============================================================
% References =================================
\usepackage{cite} 

% ===============================================================
% Font and Text Appearance ======================================
\usepackage{mathptmx}       % Altera a fonte padrão do documento para Times New Roman.

% ===============================================================
% Table Caption =================================================
\usepackage{booktabs}
\usepackage{caption} 

% ===============================================================
% Table of Contents Customization ===============================
\usepackage{tocloft}  % Oferece controle total sobre a aparência das listas de conteúdos, figuras, tabelas, etc.

% ===============================================================
% Figure Positioning ============================================
\usepackage{float}     % Melhora a interface para definir o posicionamento de objetos flutuantes como figuras e tabelas.

% ===============================================================
% Paragraph Spacing and Indentation =============================
\usepackage{setspace}  % Permite o ajuste fino do espaçamento entre linhas.
\usepackage{indentfirst} % Adiciona indentação ao primeiro parágrafo de cada seção.

% ===============================================================
% Page Layout ===================================================
\usepackage[a4paper, top=3cm, bottom=2cm, left=3cm, right=2cm]{geometry}  % Define as margens de todo o documento.
\setlength{\parindent}{4em}  % Define o tamanho da indentação para todos os parágrafos.
\setlength{\emergencystretch}{3em}

% ===============================================================
% Section Heading Customization =================================
\makeatletter
\renewcommand\paragraph{\@startsection{paragraph}{4}{\z@}%
    {2ex plus 1ex minus .2ex}%
    {1em}%
    {\normalfont\normalsize\bfseries}}
\makeatother

% ===============================================================
% Urls ==========================================================
\usepackage{hyperref} % Para criar links clicáveis
\hypersetup{
    colorlinks=true,
    linkcolor=blue,
    filecolor=magenta,
    urlcolor=blue,
    citecolor=blue,
    pdfborder={0 0 0}  % Remove o quadrado ao redor dos links
}

% ===============================================================
% Bloco de código ===============================================
\usepackage{listings}
\usepackage[utf8]{inputenc}
\lstset{ 
    inputencoding=utf8,
    extendedchars=true,
    literate={á}{{\'a}}1 {é}{{\'e}}1 {í}{{\'i}}1 {ó}{{\'o}}1 {ú}{{\'u}}1 
             {Á}{{\'A}}1 {É}{{\'E}}1 {Í}{{\'I}}1 {Ó}{{\'O}}1 {Ú}{{\'U}}1 
             {à}{{\`a}}1 {è}{{\`e}}1 {ì}{{\`i}}1 {ò}{{\`o}}1 {ù}{{\`u}}1 
             {ã}{{\~a}}1 {õ}{{\~o}}1 {â}{{\^a}}1 {ê}{{\^e}}1 {î}{{\^i}}1 
             {ô}{{\^o}}1 {û}{{\^u}}1 {ä}{{\"a}}1 {ë}{{\"e}}1 {ï}{{\"i}}1 
             {ö}{{\"o}}1 {ü}{{\"u}}1 {ç}{{\c{c}}}1 {Ç}{{\c{C}}}1
}
\usepackage{xcolor}
\usepackage{courier}
\lstset{
  backgroundcolor=\color[RGB]{249,246,239},
  basicstyle=\ttfamily\footnotesize,
  breaklines=true,
  frame=single,
  numbers=left,
  numberstyle=\tiny\color{gray},
  keywordstyle=\color[RGB]{40,40,255},
  commentstyle=\color[RGB]{0,125,0},
  stringstyle=\color[RGB]{255,0,0},
  showstringspaces=false,
  rulecolor=\color{black},
  captionpos=b,
  abovecaptionskip=5pt,
  belowcaptionskip=5pt,
  xleftmargin=0.15\textwidth,
  xrightmargin=0.15\textwidth,
  morecomment=[l]{//}
}
\renewcommand{\lstlistlistingname}{Lista de Códigos} % Comando para renomear o título da Lista de Códigos

% ===============================================================
% Begin Document ================================================
\begin{document}

% ===============================================================
% Capa ==========================================================
\input{header/capa/latex/capa.tex}

% ===============================================================
% Contra Capa ===================================================
\input{header/contra-capa/latex/contra-capa.tex}

% ===============================================================
% Resumo ========================================================
\begin{titlepage}
    \thispagestyle{empty} % Remove números de página
    \setstretch{1.5} % Espaçamento entre linhas, certifique-se de que o pacote setspace está incluído em document.tex

    \begin{center}
        \textbf{\Large RESUMO}
    \end{center}

    \vspace{1cm} % Espaço vertical

    \noindent CAGIDE FIALHO, G. Relatório do projeto de Métodos Numéricos para Equações Diferenciais II. 2024. 16 f. Trabalho da Disciplina Métodos Numéricos para Equações Diferenciais II (Graduação em Engenharia da Computação) – Universidade do Estado do Rio de Janeiro, Nova Friburgo, 2024.

    \vspace{0.4cm} % Espaço vertical

    Este trabalho analisa soluções numéricas para a equação de advecção unidimensional em um domínio com condições de contorno periódicas. Focando em métodos numéricos baseados em \textbf{TVD (Total Variation Diminishing)}, foram implementados os limitadores \textbf{Osher}, \textbf{Sweby} e \textbf{Van Albada} para avaliar sua eficiência e precisão. Os resultados das simulações numéricas foram comparados com soluções analíticas exatas, evidenciando as diferenças entre os métodos na preservação de monotonicidade, redução de oscilações e comportamento em regiões de gradiente suave. A análise destaca como os limitadores influenciam a dissipação e a dispersão, demonstrando sua aplicabilidade em problemas de transporte em fenômenos naturais e em engenharia.

    \vspace{0.4cm} % Espaço vertical

    \textbf{Palavras-chave}: Métodos Numéricos, Equação de Advecção, Métodos TVD, Simulação Numérica, Limitadores.
\end{titlepage}


% ===============================================================
% Abstract ======================================================
\begin{titlepage}
    \thispagestyle{empty} % Remove page numbers
    \setstretch{1.5} % Line spacing, make sure the setspace package is included in document.tex

    \begin{center}
        \textbf{\Large ABSTRACT}
    \end{center}

    \vspace{1cm} % Vertical space

    \noindent CAGIDE FIALHO, G. Project Report on Numerical Methods for Differential Equations II. 2024. 25 p. Course Completion Work for Numerical Methods for Differential Equations II (Bachelor’s in Computer Engineering) – Bachelor’s in Computer Engineering, State University of Rio de Janeiro, Nova Friburgo, 2024.

    \vspace{0.4cm} % Vertical space

    This work investigates numerical and analytical solutions to advection and advection-diffusion equations, utilizing Lagrangian methods and Variable Separation. The study focuses on the influence of advection and diffusion coefficients in an infinite domain, exploring how velocity and diffusion coefficient affect the dispersion and transport of solutes. Through detailed numerical simulations, the characteristics of the solutions under various conditions were examined, providing insights into the dynamics of solute in unidimensional flows. Comparative analyses between pure advection and advection-diffusion solutions highlight the significant role of diffusion in modifying the concentration profile, particularly at higher coefficients. This study not only reinforces the theoretical understanding of differential equations in modeling physical phenomena but also serves as a practical reference for engineers and scientists to apply in engineering and environmental contexts.

    \vspace{0.4cm} % Vertical space

    \textbf{Keywords}: Numerical Methods, Advection Equations, Advection-Diffusion, d’Alembert Solution, Variable Separation, Numerical Simulation.
\end{titlepage}


% ===============================================================
% Lista de figuras ==============================================
\section{Conclusão Geral}

A comparação entre os métodos numéricos para a resolução da equação de advecção unidimensional destaca diferentes compromissos entre precisão e estabilidade. O método Upwind de primeira ordem apresentou maior dissipação numérica, suavizando o perfil da solução e reduzindo a amplitude ao longo do tempo, comportamento comum em métodos de menor ordem conforme discutido por LeVeque \cite{leveque2002finite}.

Os métodos de segunda ordem Lax-Wendroff, Beam-Warming e Fromm apresentaram uma preservação superior do perfil inicial e uma dissipação reduzida. Entretanto, métodos como Lax-Wendroff e Beam-Warming introduzem pequenas oscilações nas regiões de transição, principalmente em tempos mais longos. Entre as opções de segunda ordem, o método Fromm se mostrou mais equilibrado, oferecendo boa precisão com oscilações mínimas e mantendo a estabilidade ao longo do tempo.

Esses resultados indicam que a escolha do método deve considerar o compromisso entre simplicidade e fidelidade na representação da solução. Para problemas onde a precisão é essencial, métodos de segunda ordem são recomendados, enquanto o método Upwind pode ser adequado para simulações que priorizem estabilidade e simplicidade.

% ===============================================================
% Lista de tabelas ==============================================
\section{Conclusão Geral}

A comparação entre os métodos numéricos para a resolução da equação de advecção unidimensional destaca diferentes compromissos entre precisão e estabilidade. O método Upwind de primeira ordem apresentou maior dissipação numérica, suavizando o perfil da solução e reduzindo a amplitude ao longo do tempo, comportamento comum em métodos de menor ordem conforme discutido por LeVeque \cite{leveque2002finite}.

Os métodos de segunda ordem Lax-Wendroff, Beam-Warming e Fromm apresentaram uma preservação superior do perfil inicial e uma dissipação reduzida. Entretanto, métodos como Lax-Wendroff e Beam-Warming introduzem pequenas oscilações nas regiões de transição, principalmente em tempos mais longos. Entre as opções de segunda ordem, o método Fromm se mostrou mais equilibrado, oferecendo boa precisão com oscilações mínimas e mantendo a estabilidade ao longo do tempo.

Esses resultados indicam que a escolha do método deve considerar o compromisso entre simplicidade e fidelidade na representação da solução. Para problemas onde a precisão é essencial, métodos de segunda ordem são recomendados, enquanto o método Upwind pode ser adequado para simulações que priorizem estabilidade e simplicidade.

% ===============================================================
% Lista de códigos ==============================================
\section{Conclusão Geral}

A comparação entre os métodos numéricos para a resolução da equação de advecção unidimensional destaca diferentes compromissos entre precisão e estabilidade. O método Upwind de primeira ordem apresentou maior dissipação numérica, suavizando o perfil da solução e reduzindo a amplitude ao longo do tempo, comportamento comum em métodos de menor ordem conforme discutido por LeVeque \cite{leveque2002finite}.

Os métodos de segunda ordem Lax-Wendroff, Beam-Warming e Fromm apresentaram uma preservação superior do perfil inicial e uma dissipação reduzida. Entretanto, métodos como Lax-Wendroff e Beam-Warming introduzem pequenas oscilações nas regiões de transição, principalmente em tempos mais longos. Entre as opções de segunda ordem, o método Fromm se mostrou mais equilibrado, oferecendo boa precisão com oscilações mínimas e mantendo a estabilidade ao longo do tempo.

Esses resultados indicam que a escolha do método deve considerar o compromisso entre simplicidade e fidelidade na representação da solução. Para problemas onde a precisão é essencial, métodos de segunda ordem são recomendados, enquanto o método Upwind pode ser adequado para simulações que priorizem estabilidade e simplicidade. 

% ===============================================================
% Sumário =======================================================
\section{Conclusão Geral}

A comparação entre os métodos numéricos para a resolução da equação de advecção unidimensional destaca diferentes compromissos entre precisão e estabilidade. O método Upwind de primeira ordem apresentou maior dissipação numérica, suavizando o perfil da solução e reduzindo a amplitude ao longo do tempo, comportamento comum em métodos de menor ordem conforme discutido por LeVeque \cite{leveque2002finite}.

Os métodos de segunda ordem Lax-Wendroff, Beam-Warming e Fromm apresentaram uma preservação superior do perfil inicial e uma dissipação reduzida. Entretanto, métodos como Lax-Wendroff e Beam-Warming introduzem pequenas oscilações nas regiões de transição, principalmente em tempos mais longos. Entre as opções de segunda ordem, o método Fromm se mostrou mais equilibrado, oferecendo boa precisão com oscilações mínimas e mantendo a estabilidade ao longo do tempo.

Esses resultados indicam que a escolha do método deve considerar o compromisso entre simplicidade e fidelidade na representação da solução. Para problemas onde a precisão é essencial, métodos de segunda ordem são recomendados, enquanto o método Upwind pode ser adequado para simulações que priorizem estabilidade e simplicidade.

% ===============================================================
% Introdução ====================================================
\section{Conclusão Geral}

A comparação entre os métodos numéricos para a resolução da equação de advecção unidimensional destaca diferentes compromissos entre precisão e estabilidade. O método Upwind de primeira ordem apresentou maior dissipação numérica, suavizando o perfil da solução e reduzindo a amplitude ao longo do tempo, comportamento comum em métodos de menor ordem conforme discutido por LeVeque \cite{leveque2002finite}.

Os métodos de segunda ordem Lax-Wendroff, Beam-Warming e Fromm apresentaram uma preservação superior do perfil inicial e uma dissipação reduzida. Entretanto, métodos como Lax-Wendroff e Beam-Warming introduzem pequenas oscilações nas regiões de transição, principalmente em tempos mais longos. Entre as opções de segunda ordem, o método Fromm se mostrou mais equilibrado, oferecendo boa precisão com oscilações mínimas e mantendo a estabilidade ao longo do tempo.

Esses resultados indicam que a escolha do método deve considerar o compromisso entre simplicidade e fidelidade na representação da solução. Para problemas onde a precisão é essencial, métodos de segunda ordem são recomendados, enquanto o método Upwind pode ser adequado para simulações que priorizem estabilidade e simplicidade.

% ===============================================================
% Introdução ====================================================
\section{Conclusão Geral}

A comparação entre os métodos numéricos para a resolução da equação de advecção unidimensional destaca diferentes compromissos entre precisão e estabilidade. O método Upwind de primeira ordem apresentou maior dissipação numérica, suavizando o perfil da solução e reduzindo a amplitude ao longo do tempo, comportamento comum em métodos de menor ordem conforme discutido por LeVeque \cite{leveque2002finite}.

Os métodos de segunda ordem Lax-Wendroff, Beam-Warming e Fromm apresentaram uma preservação superior do perfil inicial e uma dissipação reduzida. Entretanto, métodos como Lax-Wendroff e Beam-Warming introduzem pequenas oscilações nas regiões de transição, principalmente em tempos mais longos. Entre as opções de segunda ordem, o método Fromm se mostrou mais equilibrado, oferecendo boa precisão com oscilações mínimas e mantendo a estabilidade ao longo do tempo.

Esses resultados indicam que a escolha do método deve considerar o compromisso entre simplicidade e fidelidade na representação da solução. Para problemas onde a precisão é essencial, métodos de segunda ordem são recomendados, enquanto o método Upwind pode ser adequado para simulações que priorizem estabilidade e simplicidade.

% ===============================================================
% Desenvolvimento Teórico =======================================
\section{Conclusão Geral}

A comparação entre os métodos numéricos para a resolução da equação de advecção unidimensional destaca diferentes compromissos entre precisão e estabilidade. O método Upwind de primeira ordem apresentou maior dissipação numérica, suavizando o perfil da solução e reduzindo a amplitude ao longo do tempo, comportamento comum em métodos de menor ordem conforme discutido por LeVeque \cite{leveque2002finite}.

Os métodos de segunda ordem Lax-Wendroff, Beam-Warming e Fromm apresentaram uma preservação superior do perfil inicial e uma dissipação reduzida. Entretanto, métodos como Lax-Wendroff e Beam-Warming introduzem pequenas oscilações nas regiões de transição, principalmente em tempos mais longos. Entre as opções de segunda ordem, o método Fromm se mostrou mais equilibrado, oferecendo boa precisão com oscilações mínimas e mantendo a estabilidade ao longo do tempo.

Esses resultados indicam que a escolha do método deve considerar o compromisso entre simplicidade e fidelidade na representação da solução. Para problemas onde a precisão é essencial, métodos de segunda ordem são recomendados, enquanto o método Upwind pode ser adequado para simulações que priorizem estabilidade e simplicidade.

% ===============================================================
% Atividades ====================================================
\section{Demonstração e Análise de Resultados}

Nesta seção, apresentamos a implementação dos métodos numéricos para resolver a equação de advecção, comparando os resultados das soluções obtidas com os métodos Upwind, Lax-Wendroff, Beam-Warming e Fromm. Todos os métodos foram implementados em Python, com as bibliotecas \texttt{numpy}, \texttt{matplotlib.pyplot} e \texttt{pandas} para cálculo, visualização e manipulação de dados, respectivamente.

\begin{itemize}
    \item \textbf{Velocidade de Advecção} (\( \bar{u} \)): Representa a velocidade constante do fluxo que transporta a substância ao longo do domínio. Foi fixada como \(\bar{u} = 1.0\).
    \item \textbf{Número de Courant} (\(C\)): Definido como \(C = \frac{\bar{u} \Delta t}{\Delta x}\), onde \(\Delta t\) é o intervalo de tempo e \(\Delta x\) é o intervalo espacial. Para garantir a estabilidade dos métodos numéricos explícitos, usamos \(C = 0.8\), o que satisfaz a condição de estabilidade de Courant-Friedrichs-Lewy (CFL), \(0 \leq C \leq 1\) \cite{leveque2002finite, fletcher1991computational}.
    \item \textbf{Domínio Espacial} (\([x_{\text{min}}, x_{\text{max}}]\)): Limitado entre 0 e 1, foi discretizado em \(N = 100\) pontos para melhor resolução.
    \item \textbf{Intervalos de Tempo} (\(t=1\), \(t=3\), \(t=5\)): As simulações foram realizadas em três instantes de tempo para avaliar a evolução da concentração ao longo do domínio.
    \item \textbf{Condição Inicial} (\(c(x, 0)\)): A concentração inicial foi definida como uma função gaussiana centrada em \(x = 0.3\), somada a uma concentração uniforme entre \(x = 0.6\) e \(x = 0.8\). Esta condição inicial é dada por:
    \begin{equation}
        c(x, 0) = 1.5 \exp(-200 \cdot (x - 0.3)^2) + 
        \begin{cases}
            1.5, & \text{se } 0.6 \leq x \leq 0.8, \\
            0.0, & \text{caso contrário}.
        \end{cases}
    \end{equation}
    \item \textbf{Métodos Numéricos Implementados}:
    \begin{itemize}
        \item \textbf{Upwind}: Método de primeira ordem, que introduz dissipação e é adequado para simulações com descontinuidades.
        \item \textbf{Lax-Wendroff}: Método de segunda ordem, melhora a precisão, mas pode apresentar oscilações nas regiões de descontinuidade.
        \item \textbf{Beam-Warming}: Método de segunda ordem que reduz oscilações enquanto mantém certa dissipação.
        \item \textbf{Fromm}: Método de segunda ordem que busca equilibrar dissipação e oscilações.
    \end{itemize}
    \item \textbf{Resolução das Equações}: A função \texttt{resolverAdveccao} calcula a solução de cada método para os tempos especificados. A cada passo temporal, a densidade é atualizada e os resultados finais são armazenados para análise.
\end{itemize}

\subsection{O Método Sweby}

O método Sweby é um método do tipo TVD (Total Variation Diminishing) que utiliza um limitador para evitar oscilações não físicas em regiões de alta variação. O limitador Sweby é ajustável por meio do parâmetro \(\beta\), permitindo balancear precisão e dissipação numérica \cite{sweby1984high}.

A equação geral para o fluxo numérico utilizando o limitador Sweby é dada por:

\begin{equation}
    \phi_{\text{lim}} = \max\left(0, \min\left(\beta \theta, \min\left(1, \theta\right)\right)\right),
\end{equation}

onde \(\theta\) é o gradiente relativo entre as células adjacentes, e \(\beta\) é o parâmetro de controle do limitador, comumente usado como \(\beta = 1,5\).

A equação do método TVD aplicado ao método Sweby é expressa como:

\begin{equation}
    Q_i^{n+1} = Q_i^n - C (F_{i+1/2} - F_{i-1/2}),
\end{equation}

onde os fluxos \(F_{i+1/2}\) e \(F_{i-1/2}\) são calculados com base no limitador Sweby.

\begin{figure}[H]
    \centering
    \includegraphics[width=\textwidth]{code/images/Sweby.png}
    \caption{Solução Sweby para \(t=1\), \(t=3\) e \(t=5\), com a condição inicial representada pela linha tracejada.}
    \label{fig:sweby}
\end{figure}

\begin{table}[H]
    \centering
    \begin{tabular}{rrrrrr}
\toprule
Posicao Espacial & Condicao Inicial & Sweby t=1 & Sweby t=3 & Sweby t=5 & Posicao da Estabilidade \\
\midrule
0.000000 & 0.000000 & 0.000000 & 0.000002 & 0.000006 & 0.000000 \\
0.050000 & 0.000006 & 0.000012 & 0.000034 & 0.000053 & 0.050000 \\
0.100000 & 0.000503 & 0.000718 & 0.001223 & 0.001420 & 0.100000 \\
0.150000 & 0.016663 & 0.018961 & 0.023321 & 0.022593 & 0.150000 \\
0.200000 & 0.203003 & 0.206288 & 0.213317 & 0.192373 & 0.200000 \\
0.250000 & 0.909796 & 0.923582 & 0.967856 & 0.912867 & 0.250000 \\
0.300000 & 1.500000 & 1.437160 & 1.373804 & 1.325958 & 0.300000 \\
0.350000 & 0.909796 & 0.904268 & 0.930223 & 1.030586 & 0.350000 \\
0.400000 & 0.203003 & 0.208248 & 0.218056 & 0.261015 & 0.400000 \\
0.450000 & 0.016663 & 0.019849 & 0.025923 & 0.037466 & 0.450000 \\
0.500000 & 0.000503 & 0.000766 & 0.001917 & 0.004691 & 0.500000 \\
0.550000 & 0.000006 & 0.003678 & 0.028017 & 0.040087 & 0.550000 \\
0.600000 & 1.500000 & 0.890163 & 0.860449 & 0.740881 & 0.600000 \\
0.650000 & 1.500000 & 1.497814 & 1.475949 & 1.438465 & 0.650000 \\
0.700000 & 1.500000 & 1.500000 & 1.499627 & 1.497384 & 0.700000 \\
0.750000 & 1.500000 & 1.498258 & 1.481938 & 1.472013 & 0.750000 \\
0.800000 & 1.500000 & 0.849292 & 0.804179 & 0.905699 & 0.800000 \\
0.850000 & 0.000000 & 0.004615 & 0.036054 & 0.083578 & 0.850000 \\
0.900000 & 0.000000 & 0.000001 & 0.000308 & 0.002432 & 0.900000 \\
0.950000 & 0.000000 & 0.000000 & 0.000003 & 0.000025 & 0.950000 \\
\bottomrule
\end{tabular}

    \caption{Tabela de resultados para o método Sweby nas posições espaciais selecionadas e diferentes tempos.}
    \label{tab:sweby}
\end{table}

\subsection{Análise dos Resultados do Método Sweby}

A Figura~\ref{fig:sweby} apresenta os resultados obtidos com o método Sweby nos instantes \(t=1\), \(t=3\) e \(t=5\). Para \(t=1\), a solução mantém o perfil da condição inicial com alta precisão, preservando os gradientes e evitando oscilações. Nos tempos \(t=3\) e \(t=5\), observam-se pequenas alterações na amplitude das regiões de alta variação, porém sem oscilações significativas, evidenciando a eficácia do limitador Sweby em controlar oscilações enquanto mantém uma boa precisão.

\subsection{Implementação em Python}

O código em Python para o método Sweby utiliza a função principal \texttt{resolverAdveccaoTVD}, que aplica o limitador e calcula a evolução da densidade ao longo do tempo, conforme mostrado na Listagem~\ref{lst:codigo_sweby}. A implementação do limitador Sweby é dada por:

\begin{lstlisting}[language=Python, caption={Código para resolver a advecção usando o método Sweby}, label={lst:codigo_sweby}]
def limitadorSweby(theta, beta=1.5):
    return np.maximum(0, np.minimum(beta * theta, np.minimum(1, theta)))

def metodoTvdSweby(densidade, nt, intervaloTempo, intervaloEspacial, numeroCourant):
    """
    Método TVD para resolver a advecção utilizando o limitador de Sweby.
    """
    for n in range(nt):
        novaDensidade = densidade.copy()
        for i in range(len(densidade)):
            esquerda = (i - 1) % len(densidade)
            direita = (i + 1) % len(densidade)
            # Calcula o gradiente relativo (theta)
            theta = (densidade[i] - densidade[esquerda]) / (densidade[direita] - densidade[i] + 1e-6)
            # Fluxos para direita e esquerda
            fluxoDireita = densidade[i] + 0.5 * numeroCourant * (1 - numeroCourant) * limitadorSweby(theta) * (densidade[direita] - densidade[i])
            fluxoEsquerda = densidade[esquerda] + 0.5 * numeroCourant * (1 - numeroCourant) * limitadorSweby(theta) * (densidade[i] - densidade[esquerda])
            # Atualiza a densidade
            novaDensidade[i] = densidade[i] - numeroCourant * (fluxoDireita - fluxoEsquerda)
        densidade = novaDensidade.copy()
    return densidade
\end{lstlisting}

A implementação do método Sweby garante estabilidade e precisão ao controlar oscilações em gradientes acentuados. O número de Courant \(C=0,8\) é usado para calcular o fluxo numérico em cada interface, assegurando a convergência da solução.
 % Done
\subsection{O Método Sweby}

O método Sweby é um método do tipo TVD (Total Variation Diminishing) que utiliza um limitador para evitar oscilações não físicas em regiões de alta variação. O limitador Sweby é ajustável por meio do parâmetro \(\beta\), permitindo balancear precisão e dissipação numérica \cite{sweby1984high}.

A equação geral para o fluxo numérico utilizando o limitador Sweby é dada por:

\begin{equation}
    \phi_{\text{lim}} = \max\left(0, \min\left(\beta \theta, \min\left(1, \theta\right)\right)\right),
\end{equation}

onde \(\theta\) é o gradiente relativo entre as células adjacentes, e \(\beta\) é o parâmetro de controle do limitador, comumente usado como \(\beta = 1,5\).

A equação do método TVD aplicado ao método Sweby é expressa como:

\begin{equation}
    Q_i^{n+1} = Q_i^n - C (F_{i+1/2} - F_{i-1/2}),
\end{equation}

onde os fluxos \(F_{i+1/2}\) e \(F_{i-1/2}\) são calculados com base no limitador Sweby.

\begin{figure}[H]
    \centering
    \includegraphics[width=\textwidth]{code/images/Sweby.png}
    \caption{Solução Sweby para \(t=1\), \(t=3\) e \(t=5\), com a condição inicial representada pela linha tracejada.}
    \label{fig:sweby}
\end{figure}

\begin{table}[H]
    \centering
    \begin{tabular}{rrrrrr}
\toprule
Posicao Espacial & Condicao Inicial & Sweby t=1 & Sweby t=3 & Sweby t=5 & Posicao da Estabilidade \\
\midrule
0.000000 & 0.000000 & 0.000000 & 0.000002 & 0.000006 & 0.000000 \\
0.050000 & 0.000006 & 0.000012 & 0.000034 & 0.000053 & 0.050000 \\
0.100000 & 0.000503 & 0.000718 & 0.001223 & 0.001420 & 0.100000 \\
0.150000 & 0.016663 & 0.018961 & 0.023321 & 0.022593 & 0.150000 \\
0.200000 & 0.203003 & 0.206288 & 0.213317 & 0.192373 & 0.200000 \\
0.250000 & 0.909796 & 0.923582 & 0.967856 & 0.912867 & 0.250000 \\
0.300000 & 1.500000 & 1.437160 & 1.373804 & 1.325958 & 0.300000 \\
0.350000 & 0.909796 & 0.904268 & 0.930223 & 1.030586 & 0.350000 \\
0.400000 & 0.203003 & 0.208248 & 0.218056 & 0.261015 & 0.400000 \\
0.450000 & 0.016663 & 0.019849 & 0.025923 & 0.037466 & 0.450000 \\
0.500000 & 0.000503 & 0.000766 & 0.001917 & 0.004691 & 0.500000 \\
0.550000 & 0.000006 & 0.003678 & 0.028017 & 0.040087 & 0.550000 \\
0.600000 & 1.500000 & 0.890163 & 0.860449 & 0.740881 & 0.600000 \\
0.650000 & 1.500000 & 1.497814 & 1.475949 & 1.438465 & 0.650000 \\
0.700000 & 1.500000 & 1.500000 & 1.499627 & 1.497384 & 0.700000 \\
0.750000 & 1.500000 & 1.498258 & 1.481938 & 1.472013 & 0.750000 \\
0.800000 & 1.500000 & 0.849292 & 0.804179 & 0.905699 & 0.800000 \\
0.850000 & 0.000000 & 0.004615 & 0.036054 & 0.083578 & 0.850000 \\
0.900000 & 0.000000 & 0.000001 & 0.000308 & 0.002432 & 0.900000 \\
0.950000 & 0.000000 & 0.000000 & 0.000003 & 0.000025 & 0.950000 \\
\bottomrule
\end{tabular}

    \caption{Tabela de resultados para o método Sweby nas posições espaciais selecionadas e diferentes tempos.}
    \label{tab:sweby}
\end{table}

\subsection{Análise dos Resultados do Método Sweby}

A Figura~\ref{fig:sweby} apresenta os resultados obtidos com o método Sweby nos instantes \(t=1\), \(t=3\) e \(t=5\). Para \(t=1\), a solução mantém o perfil da condição inicial com alta precisão, preservando os gradientes e evitando oscilações. Nos tempos \(t=3\) e \(t=5\), observam-se pequenas alterações na amplitude das regiões de alta variação, porém sem oscilações significativas, evidenciando a eficácia do limitador Sweby em controlar oscilações enquanto mantém uma boa precisão.

\subsection{Implementação em Python}

O código em Python para o método Sweby utiliza a função principal \texttt{resolverAdveccaoTVD}, que aplica o limitador e calcula a evolução da densidade ao longo do tempo, conforme mostrado na Listagem~\ref{lst:codigo_sweby}. A implementação do limitador Sweby é dada por:

\begin{lstlisting}[language=Python, caption={Código para resolver a advecção usando o método Sweby}, label={lst:codigo_sweby}]
def limitadorSweby(theta, beta=1.5):
    return np.maximum(0, np.minimum(beta * theta, np.minimum(1, theta)))

def metodoTvdSweby(densidade, nt, intervaloTempo, intervaloEspacial, numeroCourant):
    """
    Método TVD para resolver a advecção utilizando o limitador de Sweby.
    """
    for n in range(nt):
        novaDensidade = densidade.copy()
        for i in range(len(densidade)):
            esquerda = (i - 1) % len(densidade)
            direita = (i + 1) % len(densidade)
            # Calcula o gradiente relativo (theta)
            theta = (densidade[i] - densidade[esquerda]) / (densidade[direita] - densidade[i] + 1e-6)
            # Fluxos para direita e esquerda
            fluxoDireita = densidade[i] + 0.5 * numeroCourant * (1 - numeroCourant) * limitadorSweby(theta) * (densidade[direita] - densidade[i])
            fluxoEsquerda = densidade[esquerda] + 0.5 * numeroCourant * (1 - numeroCourant) * limitadorSweby(theta) * (densidade[i] - densidade[esquerda])
            # Atualiza a densidade
            novaDensidade[i] = densidade[i] - numeroCourant * (fluxoDireita - fluxoEsquerda)
        densidade = novaDensidade.copy()
    return densidade
\end{lstlisting}

A implementação do método Sweby garante estabilidade e precisão ao controlar oscilações em gradientes acentuados. O número de Courant \(C=0,8\) é usado para calcular o fluxo numérico em cada interface, assegurando a convergência da solução.
 % Done
\subsection{O Método Sweby}

O método Sweby é um método do tipo TVD (Total Variation Diminishing) que utiliza um limitador para evitar oscilações não físicas em regiões de alta variação. O limitador Sweby é ajustável por meio do parâmetro \(\beta\), permitindo balancear precisão e dissipação numérica \cite{sweby1984high}.

A equação geral para o fluxo numérico utilizando o limitador Sweby é dada por:

\begin{equation}
    \phi_{\text{lim}} = \max\left(0, \min\left(\beta \theta, \min\left(1, \theta\right)\right)\right),
\end{equation}

onde \(\theta\) é o gradiente relativo entre as células adjacentes, e \(\beta\) é o parâmetro de controle do limitador, comumente usado como \(\beta = 1,5\).

A equação do método TVD aplicado ao método Sweby é expressa como:

\begin{equation}
    Q_i^{n+1} = Q_i^n - C (F_{i+1/2} - F_{i-1/2}),
\end{equation}

onde os fluxos \(F_{i+1/2}\) e \(F_{i-1/2}\) são calculados com base no limitador Sweby.

\begin{figure}[H]
    \centering
    \includegraphics[width=\textwidth]{code/images/Sweby.png}
    \caption{Solução Sweby para \(t=1\), \(t=3\) e \(t=5\), com a condição inicial representada pela linha tracejada.}
    \label{fig:sweby}
\end{figure}

\begin{table}[H]
    \centering
    \begin{tabular}{rrrrrr}
\toprule
Posicao Espacial & Condicao Inicial & Sweby t=1 & Sweby t=3 & Sweby t=5 & Posicao da Estabilidade \\
\midrule
0.000000 & 0.000000 & 0.000000 & 0.000002 & 0.000006 & 0.000000 \\
0.050000 & 0.000006 & 0.000012 & 0.000034 & 0.000053 & 0.050000 \\
0.100000 & 0.000503 & 0.000718 & 0.001223 & 0.001420 & 0.100000 \\
0.150000 & 0.016663 & 0.018961 & 0.023321 & 0.022593 & 0.150000 \\
0.200000 & 0.203003 & 0.206288 & 0.213317 & 0.192373 & 0.200000 \\
0.250000 & 0.909796 & 0.923582 & 0.967856 & 0.912867 & 0.250000 \\
0.300000 & 1.500000 & 1.437160 & 1.373804 & 1.325958 & 0.300000 \\
0.350000 & 0.909796 & 0.904268 & 0.930223 & 1.030586 & 0.350000 \\
0.400000 & 0.203003 & 0.208248 & 0.218056 & 0.261015 & 0.400000 \\
0.450000 & 0.016663 & 0.019849 & 0.025923 & 0.037466 & 0.450000 \\
0.500000 & 0.000503 & 0.000766 & 0.001917 & 0.004691 & 0.500000 \\
0.550000 & 0.000006 & 0.003678 & 0.028017 & 0.040087 & 0.550000 \\
0.600000 & 1.500000 & 0.890163 & 0.860449 & 0.740881 & 0.600000 \\
0.650000 & 1.500000 & 1.497814 & 1.475949 & 1.438465 & 0.650000 \\
0.700000 & 1.500000 & 1.500000 & 1.499627 & 1.497384 & 0.700000 \\
0.750000 & 1.500000 & 1.498258 & 1.481938 & 1.472013 & 0.750000 \\
0.800000 & 1.500000 & 0.849292 & 0.804179 & 0.905699 & 0.800000 \\
0.850000 & 0.000000 & 0.004615 & 0.036054 & 0.083578 & 0.850000 \\
0.900000 & 0.000000 & 0.000001 & 0.000308 & 0.002432 & 0.900000 \\
0.950000 & 0.000000 & 0.000000 & 0.000003 & 0.000025 & 0.950000 \\
\bottomrule
\end{tabular}

    \caption{Tabela de resultados para o método Sweby nas posições espaciais selecionadas e diferentes tempos.}
    \label{tab:sweby}
\end{table}

\subsection{Análise dos Resultados do Método Sweby}

A Figura~\ref{fig:sweby} apresenta os resultados obtidos com o método Sweby nos instantes \(t=1\), \(t=3\) e \(t=5\). Para \(t=1\), a solução mantém o perfil da condição inicial com alta precisão, preservando os gradientes e evitando oscilações. Nos tempos \(t=3\) e \(t=5\), observam-se pequenas alterações na amplitude das regiões de alta variação, porém sem oscilações significativas, evidenciando a eficácia do limitador Sweby em controlar oscilações enquanto mantém uma boa precisão.

\subsection{Implementação em Python}

O código em Python para o método Sweby utiliza a função principal \texttt{resolverAdveccaoTVD}, que aplica o limitador e calcula a evolução da densidade ao longo do tempo, conforme mostrado na Listagem~\ref{lst:codigo_sweby}. A implementação do limitador Sweby é dada por:

\begin{lstlisting}[language=Python, caption={Código para resolver a advecção usando o método Sweby}, label={lst:codigo_sweby}]
def limitadorSweby(theta, beta=1.5):
    return np.maximum(0, np.minimum(beta * theta, np.minimum(1, theta)))

def metodoTvdSweby(densidade, nt, intervaloTempo, intervaloEspacial, numeroCourant):
    """
    Método TVD para resolver a advecção utilizando o limitador de Sweby.
    """
    for n in range(nt):
        novaDensidade = densidade.copy()
        for i in range(len(densidade)):
            esquerda = (i - 1) % len(densidade)
            direita = (i + 1) % len(densidade)
            # Calcula o gradiente relativo (theta)
            theta = (densidade[i] - densidade[esquerda]) / (densidade[direita] - densidade[i] + 1e-6)
            # Fluxos para direita e esquerda
            fluxoDireita = densidade[i] + 0.5 * numeroCourant * (1 - numeroCourant) * limitadorSweby(theta) * (densidade[direita] - densidade[i])
            fluxoEsquerda = densidade[esquerda] + 0.5 * numeroCourant * (1 - numeroCourant) * limitadorSweby(theta) * (densidade[i] - densidade[esquerda])
            # Atualiza a densidade
            novaDensidade[i] = densidade[i] - numeroCourant * (fluxoDireita - fluxoEsquerda)
        densidade = novaDensidade.copy()
    return densidade
\end{lstlisting}

A implementação do método Sweby garante estabilidade e precisão ao controlar oscilações em gradientes acentuados. O número de Courant \(C=0,8\) é usado para calcular o fluxo numérico em cada interface, assegurando a convergência da solução.
 % Done
\subsection{O Método Sweby}

O método Sweby é um método do tipo TVD (Total Variation Diminishing) que utiliza um limitador para evitar oscilações não físicas em regiões de alta variação. O limitador Sweby é ajustável por meio do parâmetro \(\beta\), permitindo balancear precisão e dissipação numérica \cite{sweby1984high}.

A equação geral para o fluxo numérico utilizando o limitador Sweby é dada por:

\begin{equation}
    \phi_{\text{lim}} = \max\left(0, \min\left(\beta \theta, \min\left(1, \theta\right)\right)\right),
\end{equation}

onde \(\theta\) é o gradiente relativo entre as células adjacentes, e \(\beta\) é o parâmetro de controle do limitador, comumente usado como \(\beta = 1,5\).

A equação do método TVD aplicado ao método Sweby é expressa como:

\begin{equation}
    Q_i^{n+1} = Q_i^n - C (F_{i+1/2} - F_{i-1/2}),
\end{equation}

onde os fluxos \(F_{i+1/2}\) e \(F_{i-1/2}\) são calculados com base no limitador Sweby.

\begin{figure}[H]
    \centering
    \includegraphics[width=\textwidth]{code/images/Sweby.png}
    \caption{Solução Sweby para \(t=1\), \(t=3\) e \(t=5\), com a condição inicial representada pela linha tracejada.}
    \label{fig:sweby}
\end{figure}

\begin{table}[H]
    \centering
    \begin{tabular}{rrrrrr}
\toprule
Posicao Espacial & Condicao Inicial & Sweby t=1 & Sweby t=3 & Sweby t=5 & Posicao da Estabilidade \\
\midrule
0.000000 & 0.000000 & 0.000000 & 0.000002 & 0.000006 & 0.000000 \\
0.050000 & 0.000006 & 0.000012 & 0.000034 & 0.000053 & 0.050000 \\
0.100000 & 0.000503 & 0.000718 & 0.001223 & 0.001420 & 0.100000 \\
0.150000 & 0.016663 & 0.018961 & 0.023321 & 0.022593 & 0.150000 \\
0.200000 & 0.203003 & 0.206288 & 0.213317 & 0.192373 & 0.200000 \\
0.250000 & 0.909796 & 0.923582 & 0.967856 & 0.912867 & 0.250000 \\
0.300000 & 1.500000 & 1.437160 & 1.373804 & 1.325958 & 0.300000 \\
0.350000 & 0.909796 & 0.904268 & 0.930223 & 1.030586 & 0.350000 \\
0.400000 & 0.203003 & 0.208248 & 0.218056 & 0.261015 & 0.400000 \\
0.450000 & 0.016663 & 0.019849 & 0.025923 & 0.037466 & 0.450000 \\
0.500000 & 0.000503 & 0.000766 & 0.001917 & 0.004691 & 0.500000 \\
0.550000 & 0.000006 & 0.003678 & 0.028017 & 0.040087 & 0.550000 \\
0.600000 & 1.500000 & 0.890163 & 0.860449 & 0.740881 & 0.600000 \\
0.650000 & 1.500000 & 1.497814 & 1.475949 & 1.438465 & 0.650000 \\
0.700000 & 1.500000 & 1.500000 & 1.499627 & 1.497384 & 0.700000 \\
0.750000 & 1.500000 & 1.498258 & 1.481938 & 1.472013 & 0.750000 \\
0.800000 & 1.500000 & 0.849292 & 0.804179 & 0.905699 & 0.800000 \\
0.850000 & 0.000000 & 0.004615 & 0.036054 & 0.083578 & 0.850000 \\
0.900000 & 0.000000 & 0.000001 & 0.000308 & 0.002432 & 0.900000 \\
0.950000 & 0.000000 & 0.000000 & 0.000003 & 0.000025 & 0.950000 \\
\bottomrule
\end{tabular}

    \caption{Tabela de resultados para o método Sweby nas posições espaciais selecionadas e diferentes tempos.}
    \label{tab:sweby}
\end{table}

\subsection{Análise dos Resultados do Método Sweby}

A Figura~\ref{fig:sweby} apresenta os resultados obtidos com o método Sweby nos instantes \(t=1\), \(t=3\) e \(t=5\). Para \(t=1\), a solução mantém o perfil da condição inicial com alta precisão, preservando os gradientes e evitando oscilações. Nos tempos \(t=3\) e \(t=5\), observam-se pequenas alterações na amplitude das regiões de alta variação, porém sem oscilações significativas, evidenciando a eficácia do limitador Sweby em controlar oscilações enquanto mantém uma boa precisão.

\subsection{Implementação em Python}

O código em Python para o método Sweby utiliza a função principal \texttt{resolverAdveccaoTVD}, que aplica o limitador e calcula a evolução da densidade ao longo do tempo, conforme mostrado na Listagem~\ref{lst:codigo_sweby}. A implementação do limitador Sweby é dada por:

\begin{lstlisting}[language=Python, caption={Código para resolver a advecção usando o método Sweby}, label={lst:codigo_sweby}]
def limitadorSweby(theta, beta=1.5):
    return np.maximum(0, np.minimum(beta * theta, np.minimum(1, theta)))

def metodoTvdSweby(densidade, nt, intervaloTempo, intervaloEspacial, numeroCourant):
    """
    Método TVD para resolver a advecção utilizando o limitador de Sweby.
    """
    for n in range(nt):
        novaDensidade = densidade.copy()
        for i in range(len(densidade)):
            esquerda = (i - 1) % len(densidade)
            direita = (i + 1) % len(densidade)
            # Calcula o gradiente relativo (theta)
            theta = (densidade[i] - densidade[esquerda]) / (densidade[direita] - densidade[i] + 1e-6)
            # Fluxos para direita e esquerda
            fluxoDireita = densidade[i] + 0.5 * numeroCourant * (1 - numeroCourant) * limitadorSweby(theta) * (densidade[direita] - densidade[i])
            fluxoEsquerda = densidade[esquerda] + 0.5 * numeroCourant * (1 - numeroCourant) * limitadorSweby(theta) * (densidade[i] - densidade[esquerda])
            # Atualiza a densidade
            novaDensidade[i] = densidade[i] - numeroCourant * (fluxoDireita - fluxoEsquerda)
        densidade = novaDensidade.copy()
    return densidade
\end{lstlisting}

A implementação do método Sweby garante estabilidade e precisão ao controlar oscilações em gradientes acentuados. O número de Courant \(C=0,8\) é usado para calcular o fluxo numérico em cada interface, assegurando a convergência da solução.
 % Done
\section{Conclusão Geral}

A comparação entre os métodos numéricos para a resolução da equação de advecção unidimensional destaca diferentes compromissos entre precisão e estabilidade. O método Upwind de primeira ordem apresentou maior dissipação numérica, suavizando o perfil da solução e reduzindo a amplitude ao longo do tempo, comportamento comum em métodos de menor ordem conforme discutido por LeVeque \cite{leveque2002finite}.

Os métodos de segunda ordem Lax-Wendroff, Beam-Warming e Fromm apresentaram uma preservação superior do perfil inicial e uma dissipação reduzida. Entretanto, métodos como Lax-Wendroff e Beam-Warming introduzem pequenas oscilações nas regiões de transição, principalmente em tempos mais longos. Entre as opções de segunda ordem, o método Fromm se mostrou mais equilibrado, oferecendo boa precisão com oscilações mínimas e mantendo a estabilidade ao longo do tempo.

Esses resultados indicam que a escolha do método deve considerar o compromisso entre simplicidade e fidelidade na representação da solução. Para problemas onde a precisão é essencial, métodos de segunda ordem são recomendados, enquanto o método Upwind pode ser adequado para simulações que priorizem estabilidade e simplicidade. % Completa e Validada
\section{Conclusão Geral}

A comparação entre os métodos numéricos para a resolução da equação de advecção unidimensional destaca diferentes compromissos entre precisão e estabilidade. O método Upwind de primeira ordem apresentou maior dissipação numérica, suavizando o perfil da solução e reduzindo a amplitude ao longo do tempo, comportamento comum em métodos de menor ordem conforme discutido por LeVeque \cite{leveque2002finite}.

Os métodos de segunda ordem Lax-Wendroff, Beam-Warming e Fromm apresentaram uma preservação superior do perfil inicial e uma dissipação reduzida. Entretanto, métodos como Lax-Wendroff e Beam-Warming introduzem pequenas oscilações nas regiões de transição, principalmente em tempos mais longos. Entre as opções de segunda ordem, o método Fromm se mostrou mais equilibrado, oferecendo boa precisão com oscilações mínimas e mantendo a estabilidade ao longo do tempo.

Esses resultados indicam que a escolha do método deve considerar o compromisso entre simplicidade e fidelidade na representação da solução. Para problemas onde a precisão é essencial, métodos de segunda ordem são recomendados, enquanto o método Upwind pode ser adequado para simulações que priorizem estabilidade e simplicidade. % Completa e Validada

% ===============================================================
% Conclusão geral ===============================================
\section{Conclusão Geral}

A comparação entre os métodos numéricos para a resolução da equação de advecção unidimensional destaca diferentes compromissos entre precisão e estabilidade. O método Upwind de primeira ordem apresentou maior dissipação numérica, suavizando o perfil da solução e reduzindo a amplitude ao longo do tempo, comportamento comum em métodos de menor ordem conforme discutido por LeVeque \cite{leveque2002finite}.

Os métodos de segunda ordem Lax-Wendroff, Beam-Warming e Fromm apresentaram uma preservação superior do perfil inicial e uma dissipação reduzida. Entretanto, métodos como Lax-Wendroff e Beam-Warming introduzem pequenas oscilações nas regiões de transição, principalmente em tempos mais longos. Entre as opções de segunda ordem, o método Fromm se mostrou mais equilibrado, oferecendo boa precisão com oscilações mínimas e mantendo a estabilidade ao longo do tempo.

Esses resultados indicam que a escolha do método deve considerar o compromisso entre simplicidade e fidelidade na representação da solução. Para problemas onde a precisão é essencial, métodos de segunda ordem são recomendados, enquanto o método Upwind pode ser adequado para simulações que priorizem estabilidade e simplicidade.

% ===============================================================
% Referencias ===================================================
\section{Conclusão Geral}

A comparação entre os métodos numéricos para a resolução da equação de advecção unidimensional destaca diferentes compromissos entre precisão e estabilidade. O método Upwind de primeira ordem apresentou maior dissipação numérica, suavizando o perfil da solução e reduzindo a amplitude ao longo do tempo, comportamento comum em métodos de menor ordem conforme discutido por LeVeque \cite{leveque2002finite}.

Os métodos de segunda ordem Lax-Wendroff, Beam-Warming e Fromm apresentaram uma preservação superior do perfil inicial e uma dissipação reduzida. Entretanto, métodos como Lax-Wendroff e Beam-Warming introduzem pequenas oscilações nas regiões de transição, principalmente em tempos mais longos. Entre as opções de segunda ordem, o método Fromm se mostrou mais equilibrado, oferecendo boa precisão com oscilações mínimas e mantendo a estabilidade ao longo do tempo.

Esses resultados indicam que a escolha do método deve considerar o compromisso entre simplicidade e fidelidade na representação da solução. Para problemas onde a precisão é essencial, métodos de segunda ordem são recomendados, enquanto o método Upwind pode ser adequado para simulações que priorizem estabilidade e simplicidade.

% ===============================================================
% End Document ==================================================
\end{document}

