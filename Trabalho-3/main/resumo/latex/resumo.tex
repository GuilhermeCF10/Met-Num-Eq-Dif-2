\begin{titlepage}
    \thispagestyle{empty} % Remove números de página
    \setstretch{1.5} % Espaçamento entre linhas, certifique-se de que o pacote setspace está incluído em document.tex

    \begin{center}
        \textbf{\Large RESUMO}
    \end{center}

    \vspace{1cm} % Espaço vertical

    \noindent CAGIDE FIALHO, G. Relatório do projeto de Métodos Numéricos para Equações Diferenciais II. 2024. 16 f. Trabalho da Disciplina Métodos Numéricos para Equações Diferenciais II (Graduação em Engenharia da Computação) – Universidade do Estado do Rio de Janeiro, Nova Friburgo, 2024.

    \vspace{0.4cm} % Espaço vertical

    Este trabalho analisa soluções numéricas para a equação de advecção unidimensional em um domínio com condições de contorno periódicas. Focando em métodos numéricos baseados em \textbf{TVD (Total Variation Diminishing)}, foram implementados os limitadores \textbf{Osher}, \textbf{Sweby} e \textbf{Van Albada} para avaliar sua eficiência e precisão. Os resultados das simulações numéricas foram comparados com soluções analíticas exatas, evidenciando as diferenças entre os métodos na preservação de monotonicidade, redução de oscilações e comportamento em regiões de gradiente suave. A análise destaca como os limitadores influenciam a dissipação e a dispersão, demonstrando sua aplicabilidade em problemas de transporte em fenômenos naturais e em engenharia.

    \vspace{0.4cm} % Espaço vertical

    \textbf{Palavras-chave}: Métodos Numéricos, Equação de Advecção, Métodos TVD, Simulação Numérica, Limitadores.
\end{titlepage}
