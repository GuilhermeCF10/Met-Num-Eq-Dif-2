\section{Conclusão Geral dos Casos de Difusão}

Este estudo detalhado dos casos com coeficientes de difusão \(D = 1 \times 10^{-5}\), \(D = 0.001\), e \(D = 0.1\) revelou insights importantes sobre a interação entre advecção e difusão na dinâmica do transporte de solutos. A análise comparativa destes casos destacou como a influência da difusão na forma da distribuição de concentração é extremamente dependente da magnitude do coeficiente de difusão.

\subsection*{Influência Mínima em Baixos Coeficientes}
Nos casos com \(D = 1 \times 10^{-5}\) e \(D = 0.001\), observou-se uma estabilidade notável na forma da distribuição de concentração, que permaneceu praticamente inalterada ao longo do tempo, dominada pela advecção. Nestes cenários, a difusão mostrou-se insuficiente para modificar significativamente a dinâmica de transporte, mantendo os perfis retangulares iniciais praticamente intactos. Esses casos sublinham a limitada eficácia da difusão em baixos níveis para alterar a dispersão de solutos, ressaltando a predominância da advecção quando a difusão é marginal.

\subsection*{Transformação Significativa em Coeficiente Maior}
Em contraste, o caso com \(D = 0.1\) mostrou uma dinâmica distinta, onde a difusão teve um impacto significativo e crescente ao longo do tempo. A partir de um perfil retangular inicial similar aos casos de menor \(D\), a difusão no caso \(D = 0.1\) efetivamente suavizou e alargou o pico de concentração, divergindo consideravelmente da advecção pura. Este comportamento ilustra o poder da difusão para moldar a dinâmica de transporte em cenários onde o coeficiente de difusão alcança magnitudes mais substanciais.

\subsection*{Implicações Práticas e Teóricas}
A seleção adequada do coeficiente de difusão nos modelos de transporte é fundamental, devendo basear-se em um entendimento profundo de sua interação com a advecção. As variações notadas entre os diferentes casos sublinham a importância de uma escolha criteriosa dos parâmetros nos modelos de advecção-difusão. Esta escolha é particularmente crítica em campos como a engenharia ambiental, processamento químico e pesquisa biológica, onde prever com precisão a dispersão de contaminantes ou nutrientes pode ter implicações significativas.

Essas observações destacam a necessidade de considerar meticulosamente tanto o valor absoluto do coeficiente de difusão quanto sua relação proporcional com a advecção ao modelar sistemas de transporte de massa. Essa abordagem assegura uma representação precisa das características de dispersão, essencial tanto para simulações teóricas quanto para aplicações práticas.
