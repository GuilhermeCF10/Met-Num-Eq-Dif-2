\begin{titlepage}
    \thispagestyle{empty} % Remove números de página
    \setstretch{1.5} % Espaçamento entre linhas, certifique-se de que o pacote setspace está incluído em document.tex

    \begin{center}
        \textbf{\Large RESUMO}
    \end{center}

    \vspace{1cm} % Espaço vertical

    \noindent CAGIDE FIALHO, G. Relatório do projeto de Métodos Numéricos para Equações Diferenciais II. 2024. 25 f. Trabalho da Disciplina Métodos Numéricos para Equações Diferenciais II (Graduação em Engenharia da Computação) – Graduação em Engenharia da Computação, Universidade do Estado do Rio de Janeiro, Nova Friburgo, 2024.

    \vspace{0.4cm} % Espaço vertical

    Este trabalho investiga soluções numéricas e analíticas de equações de advecção e advecção-difusão, utilizando métodos Lagrangianos e de Separação de Variáveis. O estudo enfoca a influência dos coeficientes de advecção e difusão em um domínio infinito, explorando como a velocidade e o coeficiente de difusão afetam a dispersão e transporte de solutos. Através de simulações numéricas detalhadas, foram examinadas as características das soluções sob várias condições, proporcionando insights sobre a dinâmica do soluto em fluxos unidimensionais. As análises comparativas entre as soluções de advecção pura e advecção-difusão destacam o papel significativo da difusão em modificar o perfil de concentração, especialmente em coeficientes mais altos. Este estudo não apenas reforça a compreensão teórica das equações diferenciais na modelagem de fenômenos físicos, mas também serve como uma referência prática para engenheiros e cientistas aplicarem em contextos de engenharia e ambientais.

    \vspace{0.4cm} % Espaço vertical

    \textbf{Palavras-chave}: Métodos Numéricos, Equações de Advecção, Advecção-Difusão, Solução de d’Alembert, Separação de Variáveis, Simulação Numérica.
\end{titlepage}
