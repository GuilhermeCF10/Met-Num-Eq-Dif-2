\section{Introdução}

Este trabalho tem como objetivo investigar e comparar as soluções das equações diferenciais unidimensionais de advecção e de advecção-difusão, que são fundamentais em diversas áreas da engenharia, especialmente na modelagem de fenômenos de transporte em meios contínuos. Para isso, utilizamos duas abordagens distintas: o método Lagrangiano, baseado na solução de d’Alembert, para a equação de advecção, e o método de Separação de Variáveis, auxiliado pela função de Green, para a equação de advecção-difusão.

A equação de advecção unidimensional modela o transporte de uma substância em um meio onde a velocidade do fluxo é constante, enquanto a equação de advecção-difusão leva em consideração tanto o transporte advectivo quanto o efeito da difusão, representado por um coeficiente de difusão \( D \). Neste estudo, consideramos um domínio infinito, permitindo uma análise mais simplificada e a obtenção de soluções analíticas claras para os dois problemas.

As soluções analíticas dessas equações permitem a compreensão profunda dos fenômenos envolvidos e fornecem uma base sólida para o desenvolvimento de métodos numéricos mais avançados, que são frequentemente necessários para resolver problemas práticos com condições de contorno mais complexas. Além disso, a análise comparativa entre as soluções das duas equações, em diferentes instantes de tempo, possibilita uma avaliação detalhada do impacto do termo difusivo sobre a propagação do soluto no meio considerado.

No desenvolvimento teórico, a equação de advecção é resolvida utilizando a abordagem de d'Alembert, que explora a invariança da concentração ao longo das características do fluxo. Já a equação de advecção-difusão é abordada por meio de uma transformação que simplifica a equação para uma forma onde a solução pode ser expressa em termos de uma função de Green, proporcionando uma solução exata para o problema.

A implementação numérica das soluções, seguida de uma análise comparativa para diferentes coeficientes de difusão, permite avaliar a influência da difusão na dispersão do soluto e como essa influência se manifesta ao longo do tempo. Os resultados são apresentados em forma de tabelas e gráficos, que facilitam a visualização das diferenças entre os dois modelos e fornecem insights valiosos para a compreensão dos fenômenos de advecção e difusão.