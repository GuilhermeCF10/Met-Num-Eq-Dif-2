\section*{Caso: \( D = 0.001 \)}


% T = 0
\subsection*{Análise para o caso: \( D = 0.001 \) e \( t = 0 \)}

A Figura \ref{fig:advec_diffus_0.001_t0} mostra a distribuição da concentração de soluto no tempo inicial para um coeficiente de difusão \( D = 0.001 \). Neste caso, a diferença entre a advecção pura e a advecção-difusão é quase imperceptível, indicando que a influência da difusão é extremamente limitada neste estágio inicial. A forma retangular do perfil de concentração mantém-se praticamente inalterada, destacando o efeito mínimo da difusão quando \( D \) é substancialmente reduzido.

\begin{figure}[H]
    \centering
    \includegraphics[width=0.7\textwidth]{code/plot/Advec_Difus_t0_D0.001.png}
    \caption{Comparação das soluções de advecção e advecção-difusão para \( D = 0.001 \) no tempo \( t = 0 \). A linha tracejada representa a advecção pura, enquanto a linha contínua, quase coincidente com a advecção pura, indica a advecção-difusão.}
    \label{fig:advec_diffus_0.001_t0}
\end{figure}

\begin{table}[H]
    \centering
    \caption{Valores numéricos da concentração para \( D = 0.001 \) e \( t = 0 \)}
    \begin{tabular}{ccc}
\toprule
x & Advecção & Advecção-Difusão \\
\midrule
-2.250000 & 0.000000 & 0.000000 \\
-1.690000 & 0.000000 & 0.000000 \\
-1.140000 & 0.000000 & 0.000000 \\
-0.580000 & 1.500000 & 1.500000 \\
-0.030000 & 1.500000 & 1.500000 \\
0.530000 & 1.500000 & 1.500000 \\
1.080000 & 0.000000 & 0.000000 \\
1.640000 & 0.000000 & 0.000000 \\
2.190000 & 0.000000 & 0.000000 \\
2.750000 & 0.000000 & 0.000000 \\
\bottomrule
\end{tabular}

\end{table}

% T = 1
\subsection*{Análise para o caso: \( D = 0.001 \) e \( t = 1 \)}

A Figura \ref{fig:advec_diffus_0.001_t1} mostra a distribuição da concentração de soluto no tempo \( t = 1 \) para um coeficiente de difusão \( D = 0.001 \). Semelhante ao tempo \( t = 0 \), a difusão tem um efeito ainda muito limitado sobre a distribuição da concentração, que permanece praticamente idêntica à advecção pura. Isso demonstra que, para valores extremamente baixos de \( D \), a difusão é insuficiente para alterar significativamente o perfil inicial da concentração mesmo após um intervalo de tempo.

\begin{figure}[H]
    \centering
    \includegraphics[width=0.7\textwidth]{code/plot/Advec_Difus_t1_D0.001.png}
    \caption{Comparação das soluções de advecção e advecção-difusão para \( D = 0.001 \) no tempo \( t = 1 \). A linha tracejada representa a advecção pura, enquanto a linha contínua, quase indistinguível, indica a advecção-difusão, evidenciando o efeito mínimo da difusão.}
    \label{fig:advec_diffus_0.001_t1}
\end{figure}

\begin{table}[H]
    \centering
    \caption{Valores numéricos da concentração para \( D = 0.001 \) e \( t = 1 \)}
    \begin{tabular}{ccc}
\toprule
x & Advecção & Advecção-Difusão \\
\midrule
-0.750000 & 0.000000 & 0.000000 \\
-0.190000 & 0.000000 & 0.000000 \\
0.360000 & 0.000000 & 0.001424 \\
0.920000 & 1.500000 & 1.500000 \\
1.470000 & 1.500000 & 1.500000 \\
2.030000 & 1.500000 & 1.500000 \\
2.580000 & 0.000000 & 0.046806 \\
3.140000 & 0.000000 & 0.000000 \\
3.690000 & 0.000000 & 0.000000 \\
4.250000 & 0.000000 & 0.000000 \\
\bottomrule
\end{tabular}

\end{table}


% T = 2
\subsection*{Análise para o caso: \( D = 0.001 \) e \( t = 2 \)}

A Figura \ref{fig:advec_diffus_0.001_t2} mostra a distribuição da concentração de soluto no tempo \( t = 2 \) para um coeficiente de difusão \( D = 0.001 \). Neste momento, a diferença entre a advecção pura e a advecção-difusão continua sendo mínima, indicando que mesmo com o passar do tempo, a difusão com um coeficiente tão baixo não tem um impacto significativo na forma da distribuição da concentração. O perfil continua essencialmente retangular, destacando que a advecção domina o transporte do soluto.

\begin{figure}[H]
    \centering
    \includegraphics[width=0.7\textwidth]{code/plot/Advec_Difus_t2_D0.001.png}
    \caption{Comparação das soluções de advecção e advecção-difusão para \( D = 0.001 \) no tempo \( t = 2 \). A linha tracejada representa a advecção pura, enquanto a linha contínua, quase indistinguível da advecção, mostra o efeito ainda limitado da difusão.}
    \label{fig:advec_diffus_0.001_t2}
\end{figure}

\begin{table}[H]
    \centering
    \caption{Valores numéricos da concentração para \( D = 0.001 \) e \( t = 2 \)}
    \input{code/latex/Tabela_Advec_Difus_t2_D0.001.tex}
\end{table}

% T = 3
\subsection*{Análise para o caso: \( D = 0.001 \) e \( t = 3 \)}

A Figura \ref{fig:advec_diffus_0.001_t3} mostra a distribuição da concentração de soluto no tempo \( t = 3 \) para um coeficiente de difusão \( D = 0.001 \). Neste ponto, a advecção pura e a advecção-difusão continuam a apresentar perfis praticamente idênticos, com o perfil de concentração mantendo a forma retangular, evidenciando o efeito mínimo da difusão neste coeficiente.

\begin{figure}[H]
    \centering
    \includegraphics[width=0.7\textwidth]{code/plot/Advec_Difus_t3_D0.001.png}
    \caption{Comparação das soluções de advecção e advecção-difusão para \( D = 0.001 \) no tempo \( t = 3 \). A linha tracejada representa a advecção pura, enquanto a linha contínua, ainda quase indistinguível da advecção, ilustra a fraca influência da difusão.}
    \label{fig:advec_diffus_0.001_t3}
\end{figure}

\begin{table}[H]
    \centering
    \caption{Valores numéricos da concentração para \( D = 0.001 \) e \( t = 3 \)}
    \begin{tabular}{ccc}
\toprule
x & Advecção & Advecção-Difusão \\
\midrule
2.250000 & 0.000000 & 0.000000 \\
2.810000 & 0.000000 & 0.000000 \\
3.360000 & 0.000000 & 0.054724 \\
3.920000 & 1.500000 & 1.500000 \\
4.470000 & 1.500000 & 1.500000 \\
5.030000 & 1.500000 & 1.500000 \\
5.580000 & 0.000000 & 0.211503 \\
6.140000 & 0.000000 & 0.000000 \\
6.690000 & 0.000000 & 0.000000 \\
7.250000 & 0.000000 & 0.000000 \\
\bottomrule
\end{tabular}

\end{table}

% Falta continuar do D0.001 t4


% T = 4



% T = 5


% % Conclusão do Caso
% \subsection*{Conclusão do Caso: \( D = 0.1 \)}
% Ao longo dos intervalos observados, de \( t = 0 \) a \( t = 5 \), as soluções numéricas para a equação de advecção-difusão com um coeficiente de difusão \( D = 0.1 \) mostraram um padrão claro e consistente de alterações na distribuição da concentração de soluto. Inicialmente, em \( t = 0 \), a advecção pura e a advecção-difusão começam com o mesmo perfil retangular de concentração, mas rapidamente divergem à medida que o tempo avança.

% Com o aumento do tempo, a difusão exerce um impacto crescente na dispersão do soluto, resultando em perfis de concentração cada vez mais suavizados e alargados para a advecção-difusão. A advecção pura, por sua natureza, simplesmente translada o pulso de concentração sem alterar sua forma, enquanto a difusão, mesmo sendo moderada pelo valor relativamente pequeno de \( D = 0.1 \), efetivamente suaviza e alarga o pico de concentração, aumentando significativamente as caudas do perfil distribuído.

% Por \( t = 5 \), a diferença entre advecção pura e advecção-difusão é dramática. A distribuição sob influência da difusão mostra um pico que não apenas é mais alto e mais agudo do que nos estágios iniciais, mas também se estende muito além da localização original do pulso, demonstrando o poder da difusão em moldar a dinâmica do transporte de massa em meios contínuos.

% Esta análise ressalta a importância da difusão na modulação das características de transporte de solutos em fluxos advectivos, particularmente em contextos onde a uniformidade da dispersão é crucial para as implicações práticas, como em processos químicos, ambientais e biológicos. A escolha de \( D = 0.1 \), embora modesta, ilustra eficazmente como mesmo pequenas taxas de difusão podem influenciar significativamente a dinâmica de sistemas de transporte advectivo ao longo do tempo.
